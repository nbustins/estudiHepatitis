\documentclass[]{article}
\usepackage{lmodern}
\usepackage{amssymb,amsmath}
\usepackage{ifxetex,ifluatex}
\usepackage{fixltx2e} % provides \textsubscript
\ifnum 0\ifxetex 1\fi\ifluatex 1\fi=0 % if pdftex
  \usepackage[T1]{fontenc}
  \usepackage[utf8]{inputenc}
\else % if luatex or xelatex
  \ifxetex
    \usepackage{mathspec}
  \else
    \usepackage{fontspec}
  \fi
  \defaultfontfeatures{Ligatures=TeX,Scale=MatchLowercase}
\fi
% use upquote if available, for straight quotes in verbatim environments
\IfFileExists{upquote.sty}{\usepackage{upquote}}{}
% use microtype if available
\IfFileExists{microtype.sty}{%
\usepackage{microtype}
\UseMicrotypeSet[protrusion]{basicmath} % disable protrusion for tt fonts
}{}
\usepackage[margin=1in]{geometry}
\usepackage{hyperref}
\hypersetup{unicode=true,
            pdftitle={Pràctica 2 - Tipologia i cicle de vida de les dades},
            pdfauthor={Autor: Narcís Bustins Núñez},
            pdfborder={0 0 0},
            breaklinks=true}
\urlstyle{same}  % don't use monospace font for urls
\usepackage{color}
\usepackage{fancyvrb}
\newcommand{\VerbBar}{|}
\newcommand{\VERB}{\Verb[commandchars=\\\{\}]}
\DefineVerbatimEnvironment{Highlighting}{Verbatim}{commandchars=\\\{\}}
% Add ',fontsize=\small' for more characters per line
\usepackage{framed}
\definecolor{shadecolor}{RGB}{248,248,248}
\newenvironment{Shaded}{\begin{snugshade}}{\end{snugshade}}
\newcommand{\AlertTok}[1]{\textcolor[rgb]{0.94,0.16,0.16}{#1}}
\newcommand{\AnnotationTok}[1]{\textcolor[rgb]{0.56,0.35,0.01}{\textbf{\textit{#1}}}}
\newcommand{\AttributeTok}[1]{\textcolor[rgb]{0.77,0.63,0.00}{#1}}
\newcommand{\BaseNTok}[1]{\textcolor[rgb]{0.00,0.00,0.81}{#1}}
\newcommand{\BuiltInTok}[1]{#1}
\newcommand{\CharTok}[1]{\textcolor[rgb]{0.31,0.60,0.02}{#1}}
\newcommand{\CommentTok}[1]{\textcolor[rgb]{0.56,0.35,0.01}{\textit{#1}}}
\newcommand{\CommentVarTok}[1]{\textcolor[rgb]{0.56,0.35,0.01}{\textbf{\textit{#1}}}}
\newcommand{\ConstantTok}[1]{\textcolor[rgb]{0.00,0.00,0.00}{#1}}
\newcommand{\ControlFlowTok}[1]{\textcolor[rgb]{0.13,0.29,0.53}{\textbf{#1}}}
\newcommand{\DataTypeTok}[1]{\textcolor[rgb]{0.13,0.29,0.53}{#1}}
\newcommand{\DecValTok}[1]{\textcolor[rgb]{0.00,0.00,0.81}{#1}}
\newcommand{\DocumentationTok}[1]{\textcolor[rgb]{0.56,0.35,0.01}{\textbf{\textit{#1}}}}
\newcommand{\ErrorTok}[1]{\textcolor[rgb]{0.64,0.00,0.00}{\textbf{#1}}}
\newcommand{\ExtensionTok}[1]{#1}
\newcommand{\FloatTok}[1]{\textcolor[rgb]{0.00,0.00,0.81}{#1}}
\newcommand{\FunctionTok}[1]{\textcolor[rgb]{0.00,0.00,0.00}{#1}}
\newcommand{\ImportTok}[1]{#1}
\newcommand{\InformationTok}[1]{\textcolor[rgb]{0.56,0.35,0.01}{\textbf{\textit{#1}}}}
\newcommand{\KeywordTok}[1]{\textcolor[rgb]{0.13,0.29,0.53}{\textbf{#1}}}
\newcommand{\NormalTok}[1]{#1}
\newcommand{\OperatorTok}[1]{\textcolor[rgb]{0.81,0.36,0.00}{\textbf{#1}}}
\newcommand{\OtherTok}[1]{\textcolor[rgb]{0.56,0.35,0.01}{#1}}
\newcommand{\PreprocessorTok}[1]{\textcolor[rgb]{0.56,0.35,0.01}{\textit{#1}}}
\newcommand{\RegionMarkerTok}[1]{#1}
\newcommand{\SpecialCharTok}[1]{\textcolor[rgb]{0.00,0.00,0.00}{#1}}
\newcommand{\SpecialStringTok}[1]{\textcolor[rgb]{0.31,0.60,0.02}{#1}}
\newcommand{\StringTok}[1]{\textcolor[rgb]{0.31,0.60,0.02}{#1}}
\newcommand{\VariableTok}[1]{\textcolor[rgb]{0.00,0.00,0.00}{#1}}
\newcommand{\VerbatimStringTok}[1]{\textcolor[rgb]{0.31,0.60,0.02}{#1}}
\newcommand{\WarningTok}[1]{\textcolor[rgb]{0.56,0.35,0.01}{\textbf{\textit{#1}}}}
\usepackage{graphicx,grffile}
\makeatletter
\def\maxwidth{\ifdim\Gin@nat@width>\linewidth\linewidth\else\Gin@nat@width\fi}
\def\maxheight{\ifdim\Gin@nat@height>\textheight\textheight\else\Gin@nat@height\fi}
\makeatother
% Scale images if necessary, so that they will not overflow the page
% margins by default, and it is still possible to overwrite the defaults
% using explicit options in \includegraphics[width, height, ...]{}
\setkeys{Gin}{width=\maxwidth,height=\maxheight,keepaspectratio}
\IfFileExists{parskip.sty}{%
\usepackage{parskip}
}{% else
\setlength{\parindent}{0pt}
\setlength{\parskip}{6pt plus 2pt minus 1pt}
}
\setlength{\emergencystretch}{3em}  % prevent overfull lines
\providecommand{\tightlist}{%
  \setlength{\itemsep}{0pt}\setlength{\parskip}{0pt}}
\setcounter{secnumdepth}{0}
% Redefines (sub)paragraphs to behave more like sections
\ifx\paragraph\undefined\else
\let\oldparagraph\paragraph
\renewcommand{\paragraph}[1]{\oldparagraph{#1}\mbox{}}
\fi
\ifx\subparagraph\undefined\else
\let\oldsubparagraph\subparagraph
\renewcommand{\subparagraph}[1]{\oldsubparagraph{#1}\mbox{}}
\fi

%%% Use protect on footnotes to avoid problems with footnotes in titles
\let\rmarkdownfootnote\footnote%
\def\footnote{\protect\rmarkdownfootnote}

%%% Change title format to be more compact
\usepackage{titling}

% Create subtitle command for use in maketitle
\providecommand{\subtitle}[1]{
  \posttitle{
    \begin{center}\large#1\end{center}
    }
}

\setlength{\droptitle}{-2em}

  \title{Pràctica 2 - Tipologia i cicle de vida de les dades}
    \pretitle{\vspace{\droptitle}\centering\huge}
  \posttitle{\par}
    \author{Autor: Narcís Bustins Núñez}
    \preauthor{\centering\large\emph}
  \postauthor{\par}
      \predate{\centering\large\emph}
  \postdate{\par}
    \date{Desembre del 2019}

\usepackage{booktabs}
\usepackage{longtable}
\usepackage{array}
\usepackage{multirow}
\usepackage{wrapfig}
\usepackage{float}
\usepackage{colortbl}
\usepackage{pdflscape}
\usepackage{tabu}
\usepackage{threeparttable}
\usepackage{threeparttablex}
\usepackage[normalem]{ulem}
\usepackage{makecell}
\usepackage{xcolor}

\begin{document}
\maketitle

{
\setcounter{tocdepth}{2}
\tableofcontents
}
\hypertarget{descripciuxf3-del-dataset}{%
\section{Descripció del dataset}\label{descripciuxf3-del-dataset}}

El joc seleccionat s'ha obtingut al repositori de machine learning UCI,
el seu títol és ``Hepatitis Data Set''. Aquest dataset és molt
interessant per aquesta pràctica, ja que permetrà aplicar algorismes de
machine learining tant supervisats com no supervisats. El dataset compta
amb la classe a la qual pertany cada observació segons si ha mort o ha
sobreviscut, aquesta etiqueta permetrà aplicar algorismes supervisats.
El dataset compta amb variables tant qualitatives com quantitatives.
Això serà interessant, ja que podrem aplicar tècniques de discretització
a les variables quantitatives. És interessant tenir variables dels dos
tipus per tant de provar diferents algorismes tant els que necessitin
variables qualitatives com els q necessitin variables quantitatives.
També hi ha motius d'interès personal per haver seleccionat aquest
dataset, ja que em sembla especialment interessant com la ciència de
dades pot ajudar al camp mèdic fent estudis per tal de diagnosticar o
predir diverses malalties i així poder ajudar a molts humans. Aquest
dataset es pot utilitzar amb diversos propòsits, per exemple crear
algorismes tant de deep learining com de machine learining per ajudar a
fer el seguiment dels pacients amb hepatitis, es pot predir si el
pacient viurà o morirà, això pot ajudar a comprovar si el tractament que
s'està seguint està funcionant o no. També es poden buscar les relacions
que tenen els atributs, quines influeixen més pel diagnòstic.

Aquests són els atributs presents al dataset, els valors de les
variables qualitatives estan representats numèricament però en aquesta
descripció indicarem el significat d'aquests.\\

\begin{itemize}
\tightlist
\item
  \textbf{Class}: Classe a la qual pertany el pacient, viu o mor
  (Die/Life).
\item
  \textbf{AGE}: Indica l'edat del pacient.
\item
  \textbf{SEX}: Indica el sexe del pacient(Male/Female).
\item
  \textbf{STEROID}: Indica si el pacient ha pres esteroides (yes/no).
\item
  \textbf{ANTIVIRALS}: Indica si el pacient ha pres antivirals (yes/no).
\item
  \textbf{FATIGUE}: Indica si el pacient és sent fatigat o no (yes/no).
\item
  \textbf{MALAISE}: Indica si el pacient sent malestar (yes/no).
\item
  \textbf{ANOREXIA}: Indica si el pacient pateix anorèxia (yes/no).
\item
  \textbf{LIVER BIG}: Indica si la mida del fetge ha augmentat (yes/no).
\item
  \textbf{LIVER FIRM}: Indica si el fetge és manté ferm (yes/no).
\item
  \textbf{SPLEEN PAL}: Indica si el pacient presenta esplenomegàlia, una
  ampliació de la melsa (yes/no).
\item
  \textbf{SPIDERS}: Indica si el pacient presenta aranyes(vasos
  sanguinis engrandits) visibles.
\item
  \textbf{ASCITES}: Pesencia de líquid a la cavitat peritoneal (yes/no).
\item
  \textbf{VARICES}: Indica si el pacient presenta varius (yes/no).
\item
  \textbf{BILIRUBIN}: Indica el nivell de bilirubina del pacient.
\item
  \textbf{ALK PHOSPH}: Indica el valor de la fosfatasa alcalina del
  pacient.
\item
  \textbf{SGOT}: Valor obtingut amb l'anàlisi de sang del pacient
  després de la prova AST.
\item
  \textbf{ALBUMIN}: Indica el valor de la proteïna albúmina del pacient.
\item
  \textbf{PROTIME}: Indica el valor de la característica del pacient.\\
\item
  \textbf{HISTOLOGY}: Indica el valor de'estudiar la histología (estudis
  microscopics)(yes/no).
\end{itemize}

\hypertarget{neteja-de-les-dades}{%
\section{Neteja de les dades}\label{neteja-de-les-dades}}

Llegim el fitxer.

\begin{Shaded}
\begin{Highlighting}[]
\NormalTok{dhep <-}\StringTok{ }\KeywordTok{read.csv}\NormalTok{(}\StringTok{"data/hepatitis.csv"}\NormalTok{)}
\NormalTok{dattrs <-}\StringTok{ }\KeywordTok{c}\NormalTok{(}\StringTok{"Class"}\NormalTok{,}\StringTok{"AGE"}\NormalTok{,}\StringTok{"SEX"}\NormalTok{,}\StringTok{"STEROID"}\NormalTok{,}\StringTok{"ANTIVIRALS"}\NormalTok{,}\StringTok{"FATIGUE"}\NormalTok{,}\StringTok{"MALAISE"}\NormalTok{,}\StringTok{"ANOREXIA"}\NormalTok{,}\StringTok{"LIVER_BIG"}\NormalTok{,}\StringTok{"LIVER_FIRM"}\NormalTok{,}\StringTok{"SPLEEN_PAL"}\NormalTok{,}\StringTok{"SPIDERS"}\NormalTok{,}\StringTok{"ASCITES"}\NormalTok{,}\StringTok{"VARICES"}\NormalTok{,}\StringTok{"BILIRUBIN"}\NormalTok{,}\StringTok{"ALK_PHOSPHATE"}\NormalTok{,}\StringTok{"SGOT"}\NormalTok{,}\StringTok{"ALBUMIN"}\NormalTok{,}\StringTok{"PROTIME"}\NormalTok{,}\StringTok{"HISTOLOGY"}\NormalTok{)}

\KeywordTok{names}\NormalTok{(dhep) <-}\StringTok{ }\NormalTok{dattrs}
\end{Highlighting}
\end{Shaded}

Com podem veure el dataset compta amb 154 files amb 20 variables
(columnes).

\begin{Shaded}
\begin{Highlighting}[]
\KeywordTok{dim}\NormalTok{(dhep)}
\end{Highlighting}
\end{Shaded}

\begin{verbatim}
## [1] 154  20
\end{verbatim}

Primer mirem quin tipus s'ha assignat a cada columna. Es pot veure com
hi ha variables que s'han llegit amb el tipus erroni, haurem de corregir
el tipus assignat.

\begin{Shaded}
\begin{Highlighting}[]
\KeywordTok{str}\NormalTok{(dhep)}
\end{Highlighting}
\end{Shaded}

\begin{verbatim}
## 'data.frame':    154 obs. of  20 variables:
##  $ Class        : int  2 2 2 2 2 1 2 2 2 2 ...
##  $ AGE          : int  50 78 31 34 34 51 23 39 30 39 ...
##  $ SEX          : int  1 1 1 1 1 1 1 1 1 1 ...
##  $ STEROID      : Factor w/ 3 levels "?","1","2": 2 3 1 3 3 2 3 3 3 2 ...
##  $ ANTIVIRALS   : int  2 2 1 2 2 2 2 2 2 1 ...
##  $ FATIGUE      : Factor w/ 3 levels "?","1","2": 2 2 3 3 3 2 3 2 3 3 ...
##  $ MALAISE      : Factor w/ 3 levels "?","1","2": 3 3 3 3 3 3 3 3 3 3 ...
##  $ ANOREXIA     : Factor w/ 3 levels "?","1","2": 3 3 3 3 3 2 3 3 3 3 ...
##  $ LIVER_BIG    : Factor w/ 3 levels "?","1","2": 2 3 3 3 3 3 3 3 3 2 ...
##  $ LIVER_FIRM   : Factor w/ 3 levels "?","1","2": 3 3 3 3 3 3 3 2 3 2 ...
##  $ SPLEEN_PAL   : Factor w/ 3 levels "?","1","2": 3 3 3 3 3 2 3 3 3 3 ...
##  $ SPIDERS      : Factor w/ 3 levels "?","1","2": 3 3 3 3 3 2 3 3 3 3 ...
##  $ ASCITES      : Factor w/ 3 levels "?","1","2": 3 3 3 3 3 3 3 3 3 3 ...
##  $ VARICES      : Factor w/ 3 levels "?","1","2": 3 3 3 3 3 3 3 3 3 3 ...
##  $ BILIRUBIN    : Factor w/ 35 levels "?","0.30","0.40",..: 8 6 6 9 8 1 9 6 9 12 ...
##  $ ALK_PHOSPHATE: Factor w/ 84 levels "?","100","102",..: 19 84 51 1 83 1 1 1 1 73 ...
##  $ SGOT         : Factor w/ 85 levels "?","100","101",..: 54 48 62 31 44 1 1 60 8 46 ...
##  $ ALBUMIN      : Factor w/ 30 levels "?","2.1","2.2",..: 13 18 18 18 18 1 1 22 17 22 ...
##  $ PROTIME      : Factor w/ 45 levels "?","0","100",..: 1 1 42 1 38 1 1 1 1 44 ...
##  $ HISTOLOGY    : int  1 1 1 1 1 1 1 1 1 1 ...
\end{verbatim}

Canviem el tipus de les variables numèriques que havien estat llegides
com a factor.

\begin{Shaded}
\begin{Highlighting}[]
\NormalTok{quantattrs <-}\StringTok{ }\KeywordTok{c}\NormalTok{(}\StringTok{"AGE"}\NormalTok{,}\StringTok{"BILIRUBIN"}\NormalTok{,}\StringTok{"ALK_PHOSPHATE"}\NormalTok{,}\StringTok{"SGOT"}\NormalTok{,}\StringTok{"ALBUMIN"}\NormalTok{,}\StringTok{"PROTIME"}\NormalTok{)}
\ControlFlowTok{for}\NormalTok{ (i }\ControlFlowTok{in}\NormalTok{ quantattrs)\{}
\NormalTok{  dhep[,i] <-}\StringTok{ }\KeywordTok{as.numeric}\NormalTok{(dhep[,i])}
\NormalTok{\}}
\end{Highlighting}
\end{Shaded}

Canviem el tipus de les variables factor que han estat llegides
incorrectament.

\begin{Shaded}
\begin{Highlighting}[]
\NormalTok{factattrs <-}\StringTok{ }\KeywordTok{c}\NormalTok{(}\StringTok{"Class"}\NormalTok{,}\StringTok{"AGE"}\NormalTok{,}\StringTok{"SEX"}\NormalTok{)}
\ControlFlowTok{for}\NormalTok{ (i }\ControlFlowTok{in}\NormalTok{ factattrs)\{}
\NormalTok{  dhep[,i] <-}\StringTok{ }\KeywordTok{as.numeric}\NormalTok{(dhep[,i])}
\NormalTok{\}}
\end{Highlighting}
\end{Shaded}

Definim dos vectors amb les variables categòriques i numèriques.

\begin{Shaded}
\begin{Highlighting}[]
\NormalTok{quantattrs <-}\StringTok{ }\KeywordTok{c}\NormalTok{(}\StringTok{"AGE"}\NormalTok{,}\StringTok{"BILIRUBIN"}\NormalTok{,}\StringTok{"ALK_PHOSPHATE"}\NormalTok{,}\StringTok{"SGOT"}\NormalTok{,}\StringTok{"ALBUMIN"}\NormalTok{,}\StringTok{"PROTIME"}\NormalTok{)}
\NormalTok{catattrs <-}\StringTok{ }\KeywordTok{c}\NormalTok{(}\StringTok{"Class"}\NormalTok{,}\StringTok{"SEX"}\NormalTok{,}\StringTok{"STEROID"}\NormalTok{,}\StringTok{"ANTIVIRALS"}\NormalTok{,}\StringTok{"FATIGUE"}\NormalTok{,}\StringTok{"MALAISE"}\NormalTok{,}\StringTok{"ANOREXIA"}\NormalTok{,}\StringTok{"LIVER_BIG"}\NormalTok{,}\StringTok{"LIVER_FIRM"}\NormalTok{,}\StringTok{"SPLEEN_PAL"}\NormalTok{,}\StringTok{"SPIDERS"}\NormalTok{,}\StringTok{"ASCITES"}\NormalTok{,}\StringTok{"VARICES"}\NormalTok{,}\StringTok{"HISTOLOGY"}\NormalTok{)}
\end{Highlighting}
\end{Shaded}

\hypertarget{valors-desconeguts}{%
\subsection{Valors desconeguts}\label{valors-desconeguts}}

En la primera insepecció hem pogut veure com hi ha valors desconeguts
representats amb el signe interrogant. Mirarem quines columnes tenen
valors desconeguts.

\begin{Shaded}
\begin{Highlighting}[]
\KeywordTok{colSums}\NormalTok{(dhep }\OperatorTok{==}\StringTok{ "?"}\NormalTok{)}
\end{Highlighting}
\end{Shaded}

\begin{verbatim}
##         Class           AGE           SEX       STEROID    ANTIVIRALS 
##             0             0             0             1             0 
##       FATIGUE       MALAISE      ANOREXIA     LIVER_BIG    LIVER_FIRM 
##             1             1             1            10            11 
##    SPLEEN_PAL       SPIDERS       ASCITES       VARICES     BILIRUBIN 
##             5             5             5             5             0 
## ALK_PHOSPHATE          SGOT       ALBUMIN       PROTIME     HISTOLOGY 
##             0             0             0             0             0
\end{verbatim}

Mirem si hi ha algun valor NA. Podem veure que no.

\begin{Shaded}
\begin{Highlighting}[]
\KeywordTok{colSums}\NormalTok{(}\KeywordTok{is.na}\NormalTok{(dhep))}
\end{Highlighting}
\end{Shaded}

\begin{verbatim}
##         Class           AGE           SEX       STEROID    ANTIVIRALS 
##             0             0             0             0             0 
##       FATIGUE       MALAISE      ANOREXIA     LIVER_BIG    LIVER_FIRM 
##             0             0             0             0             0 
##    SPLEEN_PAL       SPIDERS       ASCITES       VARICES     BILIRUBIN 
##             0             0             0             0             0 
## ALK_PHOSPHATE          SGOT       ALBUMIN       PROTIME     HISTOLOGY 
##             0             0             0             0             0
\end{verbatim}

Canviarem el valor de l'interrogant per NA. Guardem en un vector tots
els atributs que contenen valors desconeguts i seguidament apliquem el
canvi.

\begin{Shaded}
\begin{Highlighting}[]
\NormalTok{missingattrs <-}\StringTok{ }\KeywordTok{names}\NormalTok{(}\KeywordTok{which}\NormalTok{(}\KeywordTok{colSums}\NormalTok{(dhep }\OperatorTok{==}\StringTok{ "?"}\NormalTok{) }\OperatorTok{>}\StringTok{ }\DecValTok{0}\NormalTok{))}

\ControlFlowTok{for}\NormalTok{ (m }\ControlFlowTok{in}\NormalTok{ missingattrs)\{}
\NormalTok{  dhep[}\KeywordTok{which}\NormalTok{(dhep[,m] }\OperatorTok{==}\StringTok{ '?'}\NormalTok{),m] <-}\StringTok{ }\OtherTok{NA}
\NormalTok{\}}
\end{Highlighting}
\end{Shaded}

El primer que farem serà imputar els valors perduts que tenim al
dataset, com que aquest dataset no és excessivament gran preferim
aproximar els valors en comptes d'eliminar les observacions que compten
amb mising values és a dir, assumirem un petit grau d'error a canvi de
mantenir més observacions. A l'hora d'aplicar algorismes supervisats és
interessant tenir grans datasets ja que es poden dividir en el dataset
de train i el de test. En el primer apartat d'anàlisi exploratori hem
comprovat com hi ha diversos atributs que tenen valors perduts. Per tal
d'imputar els valors perduts utilitzarem el mètode missForest ja que
últimament està guanyant popularitat i s'utilitza amb variables mixtes.
Encara que el kNN és un dels més populars aquest, és molt sensible a la
k que es tria.

Eliminem la columna que no volem utilitzar.

\begin{Shaded}
\begin{Highlighting}[]
\CommentTok{#dhep$PROTIME <- NULL}
\end{Highlighting}
\end{Shaded}

Imputem els valors de les variables. Carreguem la llibreria i apliquem
la funció.

\begin{Shaded}
\begin{Highlighting}[]
\KeywordTok{library}\NormalTok{(}\StringTok{"missForest"}\NormalTok{)}
\end{Highlighting}
\end{Shaded}

\begin{verbatim}
## Loading required package: randomForest
\end{verbatim}

\begin{verbatim}
## randomForest 4.6-14
\end{verbatim}

\begin{verbatim}
## Type rfNews() to see new features/changes/bug fixes.
\end{verbatim}

\begin{verbatim}
## Loading required package: foreach
\end{verbatim}

\begin{verbatim}
## Loading required package: itertools
\end{verbatim}

\begin{verbatim}
## Loading required package: iterators
\end{verbatim}

\begin{Shaded}
\begin{Highlighting}[]
\NormalTok{mf.res <-}\StringTok{ }\KeywordTok{missForest}\NormalTok{(dhep, }\DataTypeTok{variablewise =} \OtherTok{TRUE}\NormalTok{)}
\end{Highlighting}
\end{Shaded}

\begin{verbatim}
##   missForest iteration 1 in progress...done!
##   missForest iteration 2 in progress...done!
##   missForest iteration 3 in progress...done!
\end{verbatim}

Podem veure com tots els valors NA han desaparegut, han estat imputats.
També podem obtenir informació sobre l'error, es pot veure com les
columnes que no tenien cap valor per imputar tenen un error de 0, les
columnes on s'han imputat valors presenten els seus corresponents
errors, això pot afectar a l'estudi ja que els valors no són reals
simplement són suposicions.

\begin{Shaded}
\begin{Highlighting}[]
\NormalTok{dhep <-}\StringTok{ }\NormalTok{mf.res}\OperatorTok{$}\NormalTok{ximp}
\KeywordTok{colSums}\NormalTok{(}\KeywordTok{is.na}\NormalTok{(dhep))}
\end{Highlighting}
\end{Shaded}

\begin{verbatim}
##         Class           AGE           SEX       STEROID    ANTIVIRALS 
##             0             0             0             0             0 
##       FATIGUE       MALAISE      ANOREXIA     LIVER_BIG    LIVER_FIRM 
##             0             0             0             0             0 
##    SPLEEN_PAL       SPIDERS       ASCITES       VARICES     BILIRUBIN 
##             0             0             0             0             0 
## ALK_PHOSPHATE          SGOT       ALBUMIN       PROTIME     HISTOLOGY 
##             0             0             0             0             0
\end{verbatim}

\begin{Shaded}
\begin{Highlighting}[]
\NormalTok{mf.res}\OperatorTok{$}\NormalTok{OOBerror}
\end{Highlighting}
\end{Shaded}

\begin{verbatim}
##       MSE       MSE       MSE       PFC       MSE       PFC       PFC 
## 0.0000000 0.0000000 0.0000000 0.4379085 0.0000000 0.2091503 0.1960784 
##       PFC       PFC       PFC       PFC       PFC       PFC       PFC 
## 0.2352941 0.1875000 0.2937063 0.2818792 0.2416107 0.1140940 0.1275168 
##       MSE       MSE       MSE       MSE       MSE       MSE 
## 0.0000000 0.0000000 0.0000000 0.0000000 0.0000000 0.0000000
\end{verbatim}

\hypertarget{valors-extrems---outliers}{%
\subsection{Valors extrems - outliers}\label{valors-extrems---outliers}}

Mostrem una primera descripció estadística.

\begin{Shaded}
\begin{Highlighting}[]
\KeywordTok{summary}\NormalTok{(dhep)}
\end{Highlighting}
\end{Shaded}

\begin{verbatim}
##      Class            AGE             SEX        STEROID   ANTIVIRALS   
##  Min.   :1.000   Min.   : 7.00   Min.   :1.000   ?: 0    Min.   :1.000  
##  1st Qu.:2.000   1st Qu.:32.00   1st Qu.:1.000   1:75    1st Qu.:2.000  
##  Median :2.000   Median :39.00   Median :1.000   2:79    Median :2.000  
##  Mean   :1.792   Mean   :41.27   Mean   :1.097           Mean   :1.844  
##  3rd Qu.:2.000   3rd Qu.:50.00   3rd Qu.:1.000           3rd Qu.:2.000  
##  Max.   :2.000   Max.   :78.00   Max.   :2.000           Max.   :2.000  
##  FATIGUE MALAISE ANOREXIA LIVER_BIG LIVER_FIRM SPLEEN_PAL SPIDERS ASCITES
##  ?:  0   ?: 0    ?:  0    ?:  0     ?: 0       ?:  0      ?:  0   ?:  0  
##  1:100   1:61    1: 32    1: 24     1:62       1: 31      1: 52   1: 21  
##  2: 54   2:93    2:122    2:130     2:92       2:123      2:102   2:133  
##                                                                          
##                                                                          
##                                                                          
##  VARICES   BILIRUBIN     ALK_PHOSPHATE        SGOT          ALBUMIN     
##  ?:  0   Min.   : 1.00   Min.   : 1.00   Min.   : 1.00   Min.   : 1.00  
##  1: 18   1st Qu.: 6.00   1st Qu.: 8.25   1st Qu.:29.25   1st Qu.: 9.25  
##  2:136   Median : 9.00   Median :40.50   Median :47.50   Median :17.00  
##          Mean   :11.45   Mean   :40.13   Mean   :46.60   Mean   :14.70  
##          3rd Qu.:14.00   3rd Qu.:71.00   3rd Qu.:66.75   3rd Qu.:20.00  
##          Max.   :35.00   Max.   :84.00   Max.   :85.00   Max.   :30.00  
##     PROTIME        HISTOLOGY    
##  Min.   : 1.00   Min.   :1.000  
##  1st Qu.: 1.00   1st Qu.:1.000  
##  Median : 3.00   Median :1.000  
##  Mean   :13.36   Mean   :1.455  
##  3rd Qu.:25.75   3rd Qu.:2.000  
##  Max.   :45.00   Max.   :2.000
\end{verbatim}

Analitzem els valors extrems, comprovem quins atributs de tipus
quantitatiu presenten possibles valors extrems. Com veiem els atributs
AGE i BILIRUBIN tenen alguns valors extrems, en el proper apartat
valorarem si realment són tan extrems i prendrem decissions sobre com
tractar-ho.

\begin{Shaded}
\begin{Highlighting}[]
\ControlFlowTok{for}\NormalTok{ (col }\ControlFlowTok{in}\NormalTok{ quantattrs)\{}
  \KeywordTok{cat}\NormalTok{(}\StringTok{"col:"}\NormalTok{,col,}\StringTok{"Outliers:"}\NormalTok{,}\KeywordTok{length}\NormalTok{(}\KeywordTok{boxplot.stats}\NormalTok{(dhep[,col])}\OperatorTok{$}\NormalTok{out),}\StringTok{"}\CharTok{\textbackslash{}n}\StringTok{"}\NormalTok{)}
\NormalTok{\}}
\end{Highlighting}
\end{Shaded}

\begin{verbatim}
## col: AGE Outliers: 1 
## col: BILIRUBIN Outliers: 13 
## col: ALK_PHOSPHATE Outliers: 0 
## col: SGOT Outliers: 0 
## col: ALBUMIN Outliers: 0 
## col: PROTIME Outliers: 0
\end{verbatim}

Procedim a analitzar cadascun dels valors que es podria considerar
outlier.

En aquest primer cas la variable AGE representa l'edat, un valor de 78
anys no es pot considerar un outiler, mantindrem aquest valor.

\begin{Shaded}
\begin{Highlighting}[]
\KeywordTok{boxplot}\NormalTok{(dhep}\OperatorTok{$}\NormalTok{AGE)}
\end{Highlighting}
\end{Shaded}

\includegraphics{estudiHepatitis_files/figure-latex/unnamed-chunk-15-1.pdf}

\begin{Shaded}
\begin{Highlighting}[]
\KeywordTok{boxplot.stats}\NormalTok{(dhep}\OperatorTok{$}\NormalTok{AGE)}\OperatorTok{$}\NormalTok{out}
\end{Highlighting}
\end{Shaded}

\begin{verbatim}
## [1] 78
\end{verbatim}

En aquest segon cas el nivell de bilirubina és més complex. Cal fer
recerca sobre quins nivells màxims i mínims són possibles en pacient
d'hepatitis. Segons les fonts consultades els valors més elevats de
bilirubina indiquen problemes més greus, en propers apartats buscarem la
relació de la bilirubina amb la vida o mort del pacient utilitzant
testos d'estadística inferencial. Els nivells normals de bilirubina
varien en un rang d'1 fins a 1.2 (mg/dL), a partir dels 2(mg/dL) la pell
agafa un color groguenc. Sembla que aquests són valors molt elevats però
dintre un rang possible, ja que s'indica que a partir dels 30 mg/dL el
pacient es troba en estat molt crític.\\

\begin{Shaded}
\begin{Highlighting}[]
\KeywordTok{boxplot}\NormalTok{(dhep}\OperatorTok{$}\NormalTok{BILIRUBIN)}
\end{Highlighting}
\end{Shaded}

\includegraphics{estudiHepatitis_files/figure-latex/unnamed-chunk-16-1.pdf}

\begin{Shaded}
\begin{Highlighting}[]
\KeywordTok{boxplot.stats}\NormalTok{(dhep}\OperatorTok{$}\NormalTok{BILIRUBIN)}\OperatorTok{$}\NormalTok{out}
\end{Highlighting}
\end{Shaded}

\begin{verbatim}
##  [1] 32 28 30 32 33 32 27 27 32 35 29 31 34
\end{verbatim}

\begin{Shaded}
\begin{Highlighting}[]
\KeywordTok{summary}\NormalTok{(dhep}\OperatorTok{$}\NormalTok{BILIRUBIN)}
\end{Highlighting}
\end{Shaded}

\begin{verbatim}
##    Min. 1st Qu.  Median    Mean 3rd Qu.    Max. 
##    1.00    6.00    9.00   11.45   14.00   35.00
\end{verbatim}

Com hem vist els valors dels nivells de les variables categòriques no
són explicatius, actualment es troben representats amb valors numèrics,
canviarem aquests valors, ja que pot ser útil quan apliquem futurs
algorismes i vulguem extreure conclusions.

\begin{Shaded}
\begin{Highlighting}[]
\NormalTok{dhep[dhep}\OperatorTok{$}\NormalTok{Class }\OperatorTok{==}\StringTok{ }\DecValTok{1}\NormalTok{,}\StringTok{"Class"}\NormalTok{] <-}\StringTok{ 'Die'}
\NormalTok{dhep[dhep}\OperatorTok{$}\NormalTok{Class }\OperatorTok{==}\StringTok{ }\DecValTok{2}\NormalTok{,}\StringTok{"Class"}\NormalTok{] <-}\StringTok{ 'Live'}

\NormalTok{dhep[dhep}\OperatorTok{$}\NormalTok{SEX }\OperatorTok{==}\StringTok{ }\DecValTok{1}\NormalTok{,}\StringTok{"SEX"}\NormalTok{] <-}\StringTok{ 'Male'}
\NormalTok{dhep[dhep}\OperatorTok{$}\NormalTok{SEX }\OperatorTok{==}\StringTok{ }\DecValTok{2}\NormalTok{,}\StringTok{"SEX"}\NormalTok{] <-}\StringTok{ 'Female'}

\NormalTok{dhep}\OperatorTok{$}\NormalTok{SEX <-}\StringTok{ }\KeywordTok{as.factor}\NormalTok{(dhep}\OperatorTok{$}\NormalTok{SEX)}
\NormalTok{dhep}\OperatorTok{$}\NormalTok{Class <-}\StringTok{ }\KeywordTok{as.factor}\NormalTok{(dhep}\OperatorTok{$}\NormalTok{Class)}
\end{Highlighting}
\end{Shaded}

\begin{Shaded}
\begin{Highlighting}[]
\CommentTok{# Transformem a caràcter. canviem el valors dels nivells i tornem a transformar a factor, així eliminem els nivells '?' missing.}
\ControlFlowTok{for}\NormalTok{ (i }\ControlFlowTok{in}\NormalTok{ catattrs[}\OperatorTok{-}\StringTok{ }\KeywordTok{which}\NormalTok{ (catattrs }\OperatorTok\StringTok{ }\KeywordTok{list}\NormalTok{(}\StringTok{"Class"}\NormalTok{,}\StringTok{"SEX"}\NormalTok{))] )\{}
\NormalTok{  dhep[,i] <-}\StringTok{ }\KeywordTok{as.character}\NormalTok{(dhep[,i])}
  
\NormalTok{  dhep[ dhep[,i] }\OperatorTok{==}\StringTok{ }\DecValTok{1}\NormalTok{,i] <-}\StringTok{ 'Yes'}
\NormalTok{  dhep[ dhep[,i] }\OperatorTok{==}\StringTok{ }\DecValTok{2}\NormalTok{,i] <-}\StringTok{ 'No'}
  
\NormalTok{  dhep[,i] <-}\StringTok{ }\KeywordTok{as.factor}\NormalTok{(dhep[,i])}
\NormalTok{\}}
\end{Highlighting}
\end{Shaded}

\hypertarget{anuxe0lisi-de-les-dades}{%
\section{Anàlisi de les dades}\label{anuxe0lisi-de-les-dades}}

\hypertarget{seleccio-dels-grups-a-analitzar}{%
\section{Seleccio dels grups a
analitzar}\label{seleccio-dels-grups-a-analitzar}}

En aquest apartat prepararem grups que poden ser d'interés per tal
d'analitzar o comparar. En futurs apartats els utilitzarem per tal
d'extreure conclusions.

Pacients que han mort d'hepatitis.

\begin{Shaded}
\begin{Highlighting}[]
\NormalTok{dhep.die <-}\StringTok{ }\NormalTok{dhep[dhep}\OperatorTok{$}\NormalTok{Class }\OperatorTok{==}\StringTok{ "Die"}\NormalTok{,]}
\end{Highlighting}
\end{Shaded}

Pacients que han soberviscut a l'hepatitis.

\begin{Shaded}
\begin{Highlighting}[]
\NormalTok{dhep.live <-}\StringTok{ }\NormalTok{dhep[dhep}\OperatorTok{$}\NormalTok{Class }\OperatorTok{==}\StringTok{ "Live"}\NormalTok{,]}
\end{Highlighting}
\end{Shaded}

\hypertarget{proves-estaduxedstiques}{%
\subsection{Proves estadístiques}\label{proves-estaduxedstiques}}

\hypertarget{comprovaciuxf3-de-normalitat}{%
\subsubsection{Comprovació de
normalitat}\label{comprovaciuxf3-de-normalitat}}

Aplicarem un test de Shapiro-Wilk a tots els atributs quantitatius per
tal de veure en quins casos podem assumir normalitat i en quins no.
utilitzarem aquest test ja que es considera el més robust per tal de fer
la prova de normalitat. Les hipòtesis que planteja el test són les
seg¨üents:

\begin{itemize}
\tightlist
\item
  \textbf{H\_0}: La mostra \& prové d'una població amb distribució
  normal.
\item
  \textbf{H\_1}: La mostra no prové d'una població amb distribució
  normal.
\end{itemize}

Si rebutjem la hipòtesis nul·la no podrem dir que les mostres vinguin
d'una població amb distribució normal, d'altra forma si acceptem la
hipòtsis nul·la és a dir el p-valor és superior a 0.05 podrem assumir
normalitat.

Carreguem una llibreria útil per fer taules.

\begin{Shaded}
\begin{Highlighting}[]
\KeywordTok{library}\NormalTok{(}\StringTok{"kableExtra"}\NormalTok{)}
\end{Highlighting}
\end{Shaded}

Com podem veure l'única variable que segueix una distribució normal és
l'edat ``AGE''. Prendrem una conclusió conservadora i no assumirem la
normalitat de les demés variables (no aplicarem el teorema del límit
central).

\begin{Shaded}
\begin{Highlighting}[]
\NormalTok{mat <-}\StringTok{ }\OtherTok{NULL}
\NormalTok{alpha <-}\StringTok{ }\FloatTok{0.05}
\ControlFlowTok{for}\NormalTok{ (atr }\ControlFlowTok{in}\NormalTok{ quantattrs)\{}
\NormalTok{  pv <-}\StringTok{ }\KeywordTok{shapiro.test}\NormalTok{(dhep[,atr])}\OperatorTok{$}\NormalTok{p.value}
\NormalTok{  mat <-}\StringTok{ }\KeywordTok{rbind}\NormalTok{(mat, }
               \KeywordTok{c}\NormalTok{(atr,pv,}
                 \KeywordTok{ifelse}\NormalTok{(pv }\OperatorTok{>}\StringTok{ }\NormalTok{alpha,}\StringTok{"Yes"}\NormalTok{,}\StringTok{"No"}\NormalTok{))}
\NormalTok{  )}
\NormalTok{\}}

\KeywordTok{colnames}\NormalTok{(mat) <-}\StringTok{ }\KeywordTok{c}\NormalTok{(}\StringTok{"Variable"}\NormalTok{,}\StringTok{"P-Value"}\NormalTok{, }\StringTok{"Dist. Normal"}\NormalTok{)}
\KeywordTok{kable}\NormalTok{(mat) }\OperatorTok\StringTok{ }\KeywordTok{kable_styling}\NormalTok{()}
\end{Highlighting}
\end{Shaded}

\begin{table}[H]
\centering
\begin{tabular}{l|l|l}
\hline
Variable & P-Value & Dist. Normal\\
\hline
AGE & 0.103350455455303 & Yes\\
\hline
BILIRUBIN & 9.51870650701998e-12 & No\\
\hline
ALK\_PHOSPHATE & 4.00023337886688e-10 & No\\
\hline
SGOT & 0.000154910327946493 & No\\
\hline
ALBUMIN & 1.2353582877028e-06 & No\\
\hline
PROTIME & 1.72368503302654e-13 & No\\
\hline
\end{tabular}
\end{table}

\hypertarget{comprovaciuxf3-de-la-lhomoscedasticitat}{%
\subsubsection{Comprovació de la
l'homoscedasticitat}\label{comprovaciuxf3-de-la-lhomoscedasticitat}}

En aquest apartat comprovarem si Utilitzarem el test de Levene per la
variable AGE i la seva alternativa no paramètrica el test de
Fligner-Killeen per a les demés variables ja que no hem assumit
normalitat.

\hypertarget{influencia-de-les-variables-quantitatives-en-si-el-pacient-soberviuruxe0-o-moriruxe0}{%
\subsubsection{Influencia de les variables quantitatives en si el
pacient soberviurà o
morirà}\label{influencia-de-les-variables-quantitatives-en-si-el-pacient-soberviuruxe0-o-moriruxe0}}

\hypertarget{proves-paramuxe8triques}{%
\subsubsection{Proves paramètriques}\label{proves-paramuxe8triques}}

En aquest apartat aplicarem el test de t Student

\hypertarget{edat}{%
\subsubsection{Edat}\label{edat}}

En aquest cas utilitzarem un test paramètric, hem pogut assumir
normalitat després d'aplicar el test de shapiro wilk. Utilitzarem doncs
el test de t-student.\\
Com podem veure el p-valor obtingut és inferior a 0.05, rebutgem la
hipòtesi nul·la i acceptem l'alternativa. La mitjana d'edat dels
pacients que moren és superior a l'edat mitjana dels que viuen.

\begin{Shaded}
\begin{Highlighting}[]
\KeywordTok{t.test}\NormalTok{(dhep.die}\OperatorTok{$}\NormalTok{AGE,dhep.live}\OperatorTok{$}\NormalTok{AGE,}\DataTypeTok{alternative =} \StringTok{'greater'}\NormalTok{)}
\end{Highlighting}
\end{Shaded}

\begin{verbatim}
## 
##  Welch Two Sample t-test
## 
## data:  dhep.die$AGE and dhep.live$AGE
## t = 3.1862, df = 61.099, p-value = 0.001136
## alternative hypothesis: true difference in means is greater than 0
## 95 percent confidence interval:
##  3.195834      Inf
## sample estimates:
## mean of x mean of y 
##  46.59375  39.87705
\end{verbatim}

\hypertarget{bilirubina}{%
\subsubsection{Bilirubina}\label{bilirubina}}

Investigarem si la mitjana de bilirubina dels pacients de la classe DIE
és superior als pacients de la classe LIVE. Com a hipòtesi nul·la
tindrem que les dues classes tenen nivells de bilirubina iguals, com a
alternativa direm que la mitjana de bilirubina dels pacients que moren
és superior als que viuen. Utilitzarem un test no paramètric, el wilcox
test. Com hem comprovat el test de shapiro wilk no ens ha assegurat que
puguem assumir normalitat. Segons el p-value obtingut rebutgem la
hipòtesi nul·la i acceptem l'alternativa, ja que és menor a 0.05. Podem
dir que la mitjana de bilirubina dels pacients de classe DIE és superior
a la mitjana dels pacients de classe LIVE.

\begin{Shaded}
\begin{Highlighting}[]
\KeywordTok{wilcox.test}\NormalTok{(dhep.die}\OperatorTok{$}\NormalTok{BILIRUBIN,dhep.live}\OperatorTok{$}\NormalTok{BILIRUBIN, }\DataTypeTok{paired =} \OtherTok{FALSE}\NormalTok{, }\DataTypeTok{alternative =} \StringTok{'greater'}\NormalTok{)}
\end{Highlighting}
\end{Shaded}

\begin{verbatim}
## 
##  Wilcoxon rank sum test with continuity correction
## 
## data:  dhep.die$BILIRUBIN and dhep.live$BILIRUBIN
## W = 2834, p-value = 3.957e-05
## alternative hypothesis: true location shift is greater than 0
\end{verbatim}

\hypertarget{variables-qualitatives-que-influeixen-muxe9s-en-si-el-pacient-soberviuruxe0-o-moriruxe0}{%
\subsubsection{Variables qualitatives que influeixen més en si el
pacient soberviurà o
morirà}\label{variables-qualitatives-que-influeixen-muxe9s-en-si-el-pacient-soberviuruxe0-o-moriruxe0}}

En aquest apartat intentarem descobrir quines variables qualitatives
influeixen més sobre si el pacient viurà o morirà. Utilitzarem el test
de chi quadrat. Aquest test ens indica si dues variables són depenents o
independents, quan rebutjem la hipòtesis nul·la direm que hi ha una
relació entre les dues variables. Comprovarem si hi ha relació entre
totes les variables categòriques i la classe, és a dir la variable que
ens indica si el pacient viu o mor.

\begin{Shaded}
\begin{Highlighting}[]
\KeywordTok{library}\NormalTok{(}\StringTok{"ggplot2"}\NormalTok{)}
\end{Highlighting}
\end{Shaded}

\begin{verbatim}
## 
## Attaching package: 'ggplot2'
\end{verbatim}

\begin{verbatim}
## The following object is masked from 'package:randomForest':
## 
##     margin
\end{verbatim}

Aplicarem el test a totes les variables categòriques. Com podem veure a
la taula les variables FATIGUE, MALAISE, ANOREXIA, SPLEEN\_PAL, SPIDERS,
ASCITES, VARICES, HISTOLOGY.

\begin{Shaded}
\begin{Highlighting}[]
\NormalTok{mat <-}\StringTok{ }\OtherTok{NULL}
\NormalTok{alpha <-}\StringTok{ }\FloatTok{0.05}
\ControlFlowTok{for}\NormalTok{ (atr }\ControlFlowTok{in}\NormalTok{ catattrs[}\OperatorTok{-}\KeywordTok{which}\NormalTok{(catattrs }\OperatorTok\StringTok{ "Class"}\NormalTok{)])\{}
\NormalTok{  freq.table <-}\StringTok{ }\KeywordTok{table}\NormalTok{(dhep}\OperatorTok{$}\NormalTok{Class,dhep[,atr])}
  
\NormalTok{  pv <-}\StringTok{ }\KeywordTok{chisq.test}\NormalTok{(freq.table)}\OperatorTok{$}\NormalTok{p.value}
\NormalTok{  mat <-}\StringTok{ }\KeywordTok{rbind}\NormalTok{(mat, }
               \KeywordTok{c}\NormalTok{(atr,pv,}
                 \KeywordTok{ifelse}\NormalTok{(pv }\OperatorTok{<}\StringTok{ }\NormalTok{alpha,}\StringTok{"Yes"}\NormalTok{,}\StringTok{"No"}\NormalTok{))}
\NormalTok{  )}
\NormalTok{\}}
\end{Highlighting}
\end{Shaded}

\begin{verbatim}
## Warning in chisq.test(freq.table): Chi-squared approximation may be
## incorrect

## Warning in chisq.test(freq.table): Chi-squared approximation may be
## incorrect

## Warning in chisq.test(freq.table): Chi-squared approximation may be
## incorrect

## Warning in chisq.test(freq.table): Chi-squared approximation may be
## incorrect

## Warning in chisq.test(freq.table): Chi-squared approximation may be
## incorrect
\end{verbatim}

\begin{Shaded}
\begin{Highlighting}[]
\KeywordTok{colnames}\NormalTok{(mat) <-}\StringTok{ }\KeywordTok{c}\NormalTok{(}\StringTok{"Variable"}\NormalTok{,}\StringTok{"P-Value"}\NormalTok{, }\StringTok{"Relació")}
\StringTok{kable(mat) %>% kable_styling()}
\end{Highlighting}
\end{Shaded}

\begin{table}[H]
\centering
\begin{tabular}{l|l|l}
\hline
Variable & P-Value & Relació\\
\hline
SEX & 0.0796189439950327 & No\\
\hline
STEROID & 0.119734506305366 & No\\
\hline
ANTIVIRALS & 0.173247546792783 & No\\
\hline
FATIGUE & 0.000283623525807864 & Yes\\
\hline
MALAISE & 6.61626777671667e-05 & Yes\\
\hline
ANOREXIA & 0.162879415955493 & No\\
\hline
LIVER\_BIG & 0.415495439185429 & No\\
\hline
LIVER\_FIRM & 0.512590342453942 & No\\
\hline
SPLEEN\_PAL & 0.00269169673868139 & Yes\\
\hline
SPIDERS & 9.036527779379e-07 & Yes\\
\hline
ASCITES & 4.45314447847816e-09 & Yes\\
\hline
VARICES & 2.93064976157634e-05 & Yes\\
\hline
HISTOLOGY & 7.16841274432988e-05 & Yes\\
\hline
\end{tabular}
\end{table}

\hypertarget{model-supervisat}{%
\subsection{Model supervisat}\label{model-supervisat}}

En aquest apartat aplicarem la regressió logística per tal de crear un
model de classificació supervisat.

\begin{Shaded}
\begin{Highlighting}[]
\KeywordTok{library}\NormalTok{(}\StringTok{"pROC"}\NormalTok{)}
\end{Highlighting}
\end{Shaded}

\begin{verbatim}
## Type 'citation("pROC")' for a citation.
\end{verbatim}

\begin{verbatim}
## 
## Attaching package: 'pROC'
\end{verbatim}

\begin{verbatim}
## The following objects are masked from 'package:stats':
## 
##     cov, smooth, var
\end{verbatim}

\begin{Shaded}
\begin{Highlighting}[]
\KeywordTok{library}\NormalTok{(}\StringTok{"caret"}\NormalTok{)}
\end{Highlighting}
\end{Shaded}

\begin{verbatim}
## Loading required package: lattice
\end{verbatim}

\hypertarget{preparaciuxf3-del-conjunt-de-train-i-test}{%
\subsection{Preparació del conjunt de train i
test}\label{preparaciuxf3-del-conjunt-de-train-i-test}}

\begin{Shaded}
\begin{Highlighting}[]
\KeywordTok{library}\NormalTok{(}\StringTok{"rminer"}\NormalTok{) }\CommentTok{# Carreguem la llibreria que utilitzarem per dividir el dataset.}
\end{Highlighting}
\end{Shaded}

Per tal de poder avaluar els models logístics haurem de dividir el
dataset en dos, train i test. Per tal de dividir el conjunt utilitzarem
un mètode simple anomenat mètode d'exlucsió (holdout), dividirem el
conjunt total en 2/3 per al train i 1/3 per al test.

\begin{Shaded}
\begin{Highlighting}[]
\NormalTok{h<-}\KeywordTok{holdout}\NormalTok{(dhep}\OperatorTok{$}\NormalTok{Class,}\DataTypeTok{ratio=}\DecValTok{2}\OperatorTok{/}\DecValTok{3}\NormalTok{,}\DataTypeTok{mode=}\StringTok{"stratified"}\NormalTok{)}
\NormalTok{data_train<-dhep[h}\OperatorTok{$}\NormalTok{tr,]}
\NormalTok{data_test<-dhep[h}\OperatorTok{$}\NormalTok{ts,]}
\end{Highlighting}
\end{Shaded}

Podem veure com la proporció de morts i vius es manté (la diferencia és
molt petita).

\begin{Shaded}
\begin{Highlighting}[]
\NormalTok{prop.train <-}\StringTok{ }\KeywordTok{sum}\NormalTok{(data_train}\OperatorTok{$}\NormalTok{Class }\OperatorTok{==}\StringTok{ "Live"}\NormalTok{)}\OperatorTok{/}\KeywordTok{sum}\NormalTok{(data_train}\OperatorTok{$}\NormalTok{Class }\OperatorTok{==}\StringTok{ "Die"}\NormalTok{)}
\NormalTok{prop.ori <-}\StringTok{ }\KeywordTok{sum}\NormalTok{(dhep}\OperatorTok{$}\NormalTok{Class }\OperatorTok{==}\StringTok{ "Live"}\NormalTok{)}\OperatorTok{/}\KeywordTok{sum}\NormalTok{(dhep}\OperatorTok{$}\NormalTok{Class }\OperatorTok{==}\StringTok{ "Die"}\NormalTok{)}
\NormalTok{prop.test <-}\StringTok{ }\KeywordTok{sum}\NormalTok{(data_test}\OperatorTok{$}\NormalTok{Class }\OperatorTok{==}\StringTok{ "Live"}\NormalTok{)}\OperatorTok{/}\KeywordTok{sum}\NormalTok{(data_test}\OperatorTok{$}\NormalTok{Class }\OperatorTok{==}\StringTok{ "Die"}\NormalTok{)}

\KeywordTok{c}\NormalTok{(prop.ori,prop.train,prop.test)}
\end{Highlighting}
\end{Shaded}

\begin{verbatim}
## [1] 3.812500 3.857143 3.727273
\end{verbatim}

\hypertarget{regressiuxf3-loguxedstica-utilitzant-unicament-variables-quantitatives.}{%
\subsubsection{Regressió logística utilitzant unicament variables
quantitatives.}\label{regressiuxf3-loguxedstica-utilitzant-unicament-variables-quantitatives.}}

\begin{Shaded}
\begin{Highlighting}[]
\NormalTok{log.quant <-}\StringTok{ }\KeywordTok{glm}\NormalTok{(}\DataTypeTok{data =}\NormalTok{ dhep, Class}\OperatorTok{~}\StringTok{ }\NormalTok{AGE }\OperatorTok{+}\StringTok{ }\NormalTok{BILIRUBIN}\OperatorTok{+}\StringTok{ }\NormalTok{ALK_PHOSPHATE }\OperatorTok{+}\StringTok{ }\NormalTok{SGOT}\OperatorTok{+}\StringTok{ }\NormalTok{ALBUMIN}\OperatorTok{+}\StringTok{ }\NormalTok{PROTIME, }
                 \DataTypeTok{family =} \KeywordTok{binomial}\NormalTok{(}\DataTypeTok{link =} \StringTok{"logit"}\NormalTok{) )}
\KeywordTok{summary}\NormalTok{(log.quant)}
\end{Highlighting}
\end{Shaded}

\begin{verbatim}
## 
## Call:
## glm(formula = Class ~ AGE + BILIRUBIN + ALK_PHOSPHATE + SGOT + 
##     ALBUMIN + PROTIME, family = binomial(link = "logit"), data = dhep)
## 
## Deviance Residuals: 
##     Min       1Q   Median       3Q      Max  
## -2.6776   0.1906   0.2992   0.5397   2.4099  
## 
## Coefficients:
##                Estimate Std. Error z value Pr(>|z|)    
## (Intercept)    1.916611   1.130479   1.695 0.090000 .  
## AGE           -0.031282   0.019557  -1.600 0.109707    
## BILIRUBIN     -0.105696   0.027781  -3.805 0.000142 ***
## ALK_PHOSPHATE  0.001488   0.008949   0.166 0.867947    
## SGOT           0.008361   0.010196   0.820 0.412161    
## ALBUMIN        0.143233   0.041110   3.484 0.000494 ***
## PROTIME       -0.003265   0.020832  -0.157 0.875474    
## ---
## Signif. codes:  0 '***' 0.001 '**' 0.01 '*' 0.05 '.' 0.1 ' ' 1
## 
## (Dispersion parameter for binomial family taken to be 1)
## 
##     Null deviance: 157.39  on 153  degrees of freedom
## Residual deviance: 109.86  on 147  degrees of freedom
## AIC: 123.86
## 
## Number of Fisher Scoring iterations: 5
\end{verbatim}

Podem veure com utilitzant simplement els atributs quantitatius obtenim
un model bastant bo amb un AUC de 0.825.

\begin{Shaded}
\begin{Highlighting}[]
\NormalTok{prediction.quant <-}\StringTok{ }\KeywordTok{predict}\NormalTok{(log.quant, data_test, }\DataTypeTok{type =} \StringTok{"response"}\NormalTok{)}

\NormalTok{pROC.quant <-}\StringTok{ }\KeywordTok{roc}\NormalTok{(data_test}\OperatorTok{$}\NormalTok{Class, prediction.quant, }\DataTypeTok{plot =} \OtherTok{TRUE}\NormalTok{, }\DataTypeTok{print.auc =} \OtherTok{TRUE}\NormalTok{)}
\end{Highlighting}
\end{Shaded}

\begin{verbatim}
## Setting levels: control = Die, case = Live
\end{verbatim}

\begin{verbatim}
## Setting direction: controls < cases
\end{verbatim}

\includegraphics{estudiHepatitis_files/figure-latex/unnamed-chunk-32-1.pdf}

\begin{Shaded}
\begin{Highlighting}[]
\KeywordTok{coords}\NormalTok{(pROC.quant, }\StringTok{"best"}\NormalTok{, }\StringTok{"threshold"}\NormalTok{)}
\end{Highlighting}
\end{Shaded}

\begin{verbatim}
## Warning in coords.roc(pROC.quant, "best", "threshold"): An upcoming
## version of pROC will set the 'transpose' argument to FALSE by default.
## Set transpose = TRUE explicitly to keep the current behavior, or
## transpose = FALSE to adopt the new one and silence this warning. Type
## help(coords_transpose) for additional information.
\end{verbatim}

\begin{verbatim}
##   threshold specificity sensitivity 
##   0.6743504   0.8181818   0.7804878
\end{verbatim}

\hypertarget{regressiuxf3-loguxedstica-utilitzant-unicament-variables-qualitatives.}{%
\subsubsection{Regressió logística utilitzant unicament variables
qualitatives.}\label{regressiuxf3-loguxedstica-utilitzant-unicament-variables-qualitatives.}}

En aquest apartat utilitzarem les variables qualitatives que hem
determinat, tenen efecte sobre si el pacient viurà o morirà utilitzant
la prova de chi quadrat. Observem el p-valor que s'obté, awuest
correspon al test de Wald i ens indica si el coeficient és
significativament diferent de 0. En aquest cas eliminarem les variables
amb un p-valor superior a 0.10.

\begin{Shaded}
\begin{Highlighting}[]
\NormalTok{log.qual <-}\StringTok{ }\KeywordTok{glm}\NormalTok{(}\DataTypeTok{data =}\NormalTok{ dhep, Class}\OperatorTok{~}\StringTok{ }\NormalTok{FATIGUE}\OperatorTok{+}\NormalTok{MALAISE}\OperatorTok{+}\NormalTok{SPLEEN_PAL}\OperatorTok{+}\NormalTok{SPIDERS}\OperatorTok{+}\NormalTok{ASCITES}\OperatorTok{+}\NormalTok{VARICES}\OperatorTok{+}\NormalTok{HISTOLOGY,}
                 \DataTypeTok{family =} \KeywordTok{binomial}\NormalTok{(}\DataTypeTok{link =} \StringTok{"logit"}\NormalTok{) )}
\KeywordTok{summary}\NormalTok{(log.qual)}
\end{Highlighting}
\end{Shaded}

\begin{verbatim}
## 
## Call:
## glm(formula = Class ~ FATIGUE + MALAISE + SPLEEN_PAL + SPIDERS + 
##     ASCITES + VARICES + HISTOLOGY, family = binomial(link = "logit"), 
##     data = dhep)
## 
## Deviance Residuals: 
##     Min       1Q   Median       3Q      Max  
## -2.5934   0.1971   0.2670   0.4350   2.0046  
## 
## Coefficients:
##               Estimate Std. Error z value Pr(>|z|)    
## (Intercept)     3.3275     0.8241   4.038  5.4e-05 ***
## FATIGUEYes     -0.6149     0.9544  -0.644  0.51938    
## MALAISEYes     -1.0064     0.6446  -1.561  0.11845    
## SPLEEN_PALYes  -0.9268     0.5829  -1.590  0.11183    
## SPIDERSYes     -0.9864     0.5576  -1.769  0.07690 .  
## ASCITESYes     -1.6582     0.6340  -2.616  0.00891 ** 
## VARICESYes     -0.7559     0.6615  -1.143  0.25315    
## HISTOLOGYYes    0.6040     0.5974   1.011  0.31199    
## ---
## Signif. codes:  0 '***' 0.001 '**' 0.01 '*' 0.05 '.' 0.1 ' ' 1
## 
## (Dispersion parameter for binomial family taken to be 1)
## 
##     Null deviance: 157.39  on 153  degrees of freedom
## Residual deviance: 101.81  on 146  degrees of freedom
## AIC: 117.81
## 
## Number of Fisher Scoring iterations: 6
\end{verbatim}

\begin{Shaded}
\begin{Highlighting}[]
\NormalTok{prediction.qual <-}\StringTok{ }\KeywordTok{predict}\NormalTok{(log.qual, data_test, }\DataTypeTok{type =} \StringTok{"response"}\NormalTok{)}

\NormalTok{pROC.qual <-}\StringTok{ }\KeywordTok{roc}\NormalTok{(data_test}\OperatorTok{$}\NormalTok{Class, prediction.qual, }\DataTypeTok{plot =} \OtherTok{TRUE}\NormalTok{, }\DataTypeTok{print.auc =} \OtherTok{TRUE}\NormalTok{)}
\end{Highlighting}
\end{Shaded}

\begin{verbatim}
## Setting levels: control = Die, case = Live
\end{verbatim}

\begin{verbatim}
## Setting direction: controls < cases
\end{verbatim}

\includegraphics{estudiHepatitis_files/figure-latex/unnamed-chunk-34-1.pdf}

\begin{Shaded}
\begin{Highlighting}[]
\KeywordTok{coords}\NormalTok{(pROC.qual, }\StringTok{"best"}\NormalTok{, }\StringTok{"threshold"}\NormalTok{)}
\end{Highlighting}
\end{Shaded}

\begin{verbatim}
## Warning in coords.roc(pROC.qual, "best", "threshold"): An upcoming
## version of pROC will set the 'transpose' argument to FALSE by default.
## Set transpose = TRUE explicitly to keep the current behavior, or
## transpose = FALSE to adopt the new one and silence this warning. Type
## help(coords_transpose) for additional information.
\end{verbatim}

\begin{verbatim}
##   threshold specificity sensitivity 
##   0.8381383   0.9090909   0.8292683
\end{verbatim}

\hypertarget{reducciuxf3-del-model}{%
\paragraph{Reducció del model}\label{reducciuxf3-del-model}}

Preparem el model reduit, hem eliminat les variables que tenien un
p-valor superior a 0.10 al test de Wald i ens hem quedat amb SPIDERS i
ASCITES.

\begin{Shaded}
\begin{Highlighting}[]
\NormalTok{log.qual.red <-}\StringTok{ }\KeywordTok{glm}\NormalTok{(}\DataTypeTok{data =}\NormalTok{ dhep, Class}\OperatorTok{~}\NormalTok{SPIDERS}\OperatorTok{+}\NormalTok{ASCITES,}
                 \DataTypeTok{family =} \KeywordTok{binomial}\NormalTok{(}\DataTypeTok{link =} \StringTok{"logit"}\NormalTok{) )}
\KeywordTok{summary}\NormalTok{(log.qual.red)}
\end{Highlighting}
\end{Shaded}

\begin{verbatim}
## 
## Call:
## glm(formula = Class ~ SPIDERS + ASCITES, family = binomial(link = "logit"), 
##     data = dhep)
## 
## Deviance Residuals: 
##     Min       1Q   Median       3Q      Max  
## -2.3321   0.3693   0.3693   0.3693   1.8575  
## 
## Coefficients:
##             Estimate Std. Error z value Pr(>|z|)    
## (Intercept)   2.6512     0.3862   6.864 6.69e-12 ***
## SPIDERSYes   -1.7483     0.4875  -3.587 0.000335 ***
## ASCITESYes   -2.4319     0.5884  -4.133 3.58e-05 ***
## ---
## Signif. codes:  0 '***' 0.001 '**' 0.01 '*' 0.05 '.' 0.1 ' ' 1
## 
## (Dispersion parameter for binomial family taken to be 1)
## 
##     Null deviance: 157.39  on 153  degrees of freedom
## Residual deviance: 113.40  on 151  degrees of freedom
## AIC: 119.4
## 
## Number of Fisher Scoring iterations: 5
\end{verbatim}

Obtenim uns molt bons resultats, l'AUC s'ha mantingut molt elevat amb un
valor de 0.812 i hem reduït el model utilitzant només dues variables.
Naturalment l'AUC ha disminuit ja que les demés variables aportaven una
mica d'informació encara que no fos extremadament significativa però amb
aquesta versió reduïda hem aconseguit un model molt més petit sense
comprometre en gran mesura la capacitat predictiva.

\begin{Shaded}
\begin{Highlighting}[]
\NormalTok{prediction.qual.red <-}\StringTok{ }\KeywordTok{predict}\NormalTok{(log.qual.red, data_test, }\DataTypeTok{type =} \StringTok{"response"}\NormalTok{)}

\NormalTok{pROC.qual.red <-}\StringTok{ }\KeywordTok{roc}\NormalTok{(data_test}\OperatorTok{$}\NormalTok{Class, prediction.qual.red, }\DataTypeTok{plot =} \OtherTok{TRUE}\NormalTok{, }\DataTypeTok{print.auc =} \OtherTok{TRUE}\NormalTok{)}
\end{Highlighting}
\end{Shaded}

\begin{verbatim}
## Setting levels: control = Die, case = Live
\end{verbatim}

\begin{verbatim}
## Setting direction: controls < cases
\end{verbatim}

\includegraphics{estudiHepatitis_files/figure-latex/unnamed-chunk-36-1.pdf}

\begin{Shaded}
\begin{Highlighting}[]
\KeywordTok{coords}\NormalTok{(pROC.qual.red, }\StringTok{"best"}\NormalTok{, }\StringTok{"threshold"}\NormalTok{)}
\end{Highlighting}
\end{Shaded}

\begin{verbatim}
## Warning in coords.roc(pROC.qual.red, "best", "threshold"): An upcoming
## version of pROC will set the 'transpose' argument to FALSE by default.
## Set transpose = TRUE explicitly to keep the current behavior, or
## transpose = FALSE to adopt the new one and silence this warning. Type
## help(coords_transpose) for additional information.
\end{verbatim}

\begin{verbatim}
##   threshold specificity sensitivity 
##   0.8228230   0.9090909   0.7560976
\end{verbatim}

\hypertarget{regressiuxf3-loguxedstica-utilitzant-variables-mixtes.}{%
\subsubsection{Regressió logística utilitzant variables
mixtes.}\label{regressiuxf3-loguxedstica-utilitzant-variables-mixtes.}}

En aquest ultim model utilitzarem les variables tant qualitatives com
quantitatives que hem vist que tenien més efecte en els models. Veiem
que totes les variables aporten, segons el test de Wald els coeficients
que obtenim són significativament diferents de 0.

\begin{Shaded}
\begin{Highlighting}[]
\NormalTok{log.mixt<-}\StringTok{ }\KeywordTok{glm}\NormalTok{(}\DataTypeTok{data =}\NormalTok{ dhep, Class}\OperatorTok{~}\NormalTok{SPIDERS}\OperatorTok{+}\NormalTok{ASCITES}\OperatorTok{+}\NormalTok{BILIRUBIN}\OperatorTok{+}\NormalTok{ALBUMIN,}
                 \DataTypeTok{family =} \KeywordTok{binomial}\NormalTok{(}\DataTypeTok{link =} \StringTok{"logit"}\NormalTok{) )}
\KeywordTok{summary}\NormalTok{(log.mixt)}
\end{Highlighting}
\end{Shaded}

\begin{verbatim}
## 
## Call:
## glm(formula = Class ~ SPIDERS + ASCITES + BILIRUBIN + ALBUMIN, 
##     family = binomial(link = "logit"), data = dhep)
## 
## Deviance Residuals: 
##     Min       1Q   Median       3Q      Max  
## -2.8150   0.1363   0.2057   0.4979   1.6758  
## 
## Coefficients:
##             Estimate Std. Error z value Pr(>|z|)    
## (Intercept)  1.93419    0.68635   2.818 0.004831 ** 
## SPIDERSYes  -1.77657    0.56298  -3.156 0.001601 ** 
## ASCITESYes  -1.67104    0.63659  -2.625 0.008665 ** 
## BILIRUBIN   -0.08176    0.02969  -2.753 0.005897 ** 
## ALBUMIN      0.14088    0.04113   3.426 0.000614 ***
## ---
## Signif. codes:  0 '***' 0.001 '**' 0.01 '*' 0.05 '.' 0.1 ' ' 1
## 
## (Dispersion parameter for binomial family taken to be 1)
## 
##     Null deviance: 157.393  on 153  degrees of freedom
## Residual deviance:  90.992  on 149  degrees of freedom
## AIC: 100.99
## 
## Number of Fisher Scoring iterations: 6
\end{verbatim}

Podem veure com l'AUC ha augmentat lleugerament, tenint en compte els
nivells obtinguts que ja eren molt bons hem aconseguit una millora
utilitzant variables mixtes.

\begin{Shaded}
\begin{Highlighting}[]
\NormalTok{prediction.mixt <-}\StringTok{ }\KeywordTok{predict}\NormalTok{(log.mixt, data_test, }\DataTypeTok{type =} \StringTok{"response"}\NormalTok{)}

\NormalTok{pROC.mixt <-}\StringTok{ }\KeywordTok{roc}\NormalTok{(data_test}\OperatorTok{$}\NormalTok{Class, prediction.mixt, }\DataTypeTok{plot =} \OtherTok{TRUE}\NormalTok{, }\DataTypeTok{print.auc =} \OtherTok{TRUE}\NormalTok{)}
\end{Highlighting}
\end{Shaded}

\begin{verbatim}
## Setting levels: control = Die, case = Live
\end{verbatim}

\begin{verbatim}
## Setting direction: controls < cases
\end{verbatim}

\includegraphics{estudiHepatitis_files/figure-latex/unnamed-chunk-38-1.pdf}

\begin{Shaded}
\begin{Highlighting}[]
\KeywordTok{coords}\NormalTok{(pROC.mixt, }\StringTok{"best"}\NormalTok{, }\StringTok{"threshold"}\NormalTok{)}
\end{Highlighting}
\end{Shaded}

\begin{verbatim}
## Warning in coords.roc(pROC.mixt, "best", "threshold"): An upcoming
## version of pROC will set the 'transpose' argument to FALSE by default.
## Set transpose = TRUE explicitly to keep the current behavior, or
## transpose = FALSE to adopt the new one and silence this warning. Type
## help(coords_transpose) for additional information.
\end{verbatim}

\begin{verbatim}
##   threshold specificity sensitivity 
##   0.8141618   0.9090909   0.7317073
\end{verbatim}


\end{document}
